\documentclass[a4paper,12pt]{report}

\usepackage{cmap}
\usepackage[T2A]{fontenc}
\usepackage[utf8]{inputenc}
\usepackage[russian]{babel}
\usepackage{amsmath,amsfonts,amssymb}
\usepackage{graphicx}
\usepackage{sidecap}
\usepackage{wrapfig}
\usepackage{indentfirst}

\begin{document} 

\begin{titlepage} 

\begin{center} 

\large Федеральное государственное автономное образовательное учреждение высшего образования «Санкт-Петербургский государственный электротехнический университет «ЛЭТИ» им. В.И. Ульянова (Ленина)»\\
кафедра Вычислительной техники\\[5cm] 

\huge ОТЧЕТ\\ по лабораторной работе № 7\\[0.5cm] 
\large <<Указатели на структуры и функции>>\\[3.7cm]

\begin{minipage}{1\textwidth}
    \begin{flushleft}
        \emph{Автор:} Стукен В.А.\\
        \emph{Группа:} 2307\\
        \emph{Факультет:} ФКТИ\\
        \emph{Преподаватель:} Аббас Саддам Ахмед\\
    \end{flushleft}
\end{minipage}

\vfill

Санкт-Петербург, 2023\\
{\large \LaTeX}

\end{center}
\thispagestyle{empty}
\end{titlepage}

\section*{Задание(вариант 13)}
Выбор записей, в которых значение любого элемента поля с числовым массивом (выбор из меню) совпадает с заданным, сортировка результата по возрастанию значений любого из элементов поля с числовым массивом (выбор признака сортировки — из меню).

\section*{Постановка задачи и описание решения}
\par

Сначала инициализуирем структуру my_computer с соответствующими полями. Далее из файла заполняем структуру данных.
Далее выводим меню пользователя, где он может 
\begin{itemize}
    \item добавить новую запись в структуре через ввод с клавиатуры
    \item Найти определенное значение в числовых массивах структуры
    \item Отсортировать структуру по определенному параметру
    \item Показать все записи структуры
\end{itemize}

Рассмотрим каждый из пунктов подробнее:\\
\begin{itemize}
    \item Добавление новой записи:\\
        Вызываем функцию add_new_el_to_struct(), в которой мы сначала выделяем память под новую запись в структуре, далее читаем ввод с клавиатуры и заполняем соответствующие поля структуры.
    \item Найти определенное значение в числовых массивах:
        Спрашиваем у пользователя в каком из двух массивов искать введенное с клавиатуры значение. Проходимся по всем записям и соответствующим массивам в них, и если находим это значение, то выводим эту запись полностью.
    \item Отсортировать записи структуры по определенному параметру:
        Просим пользователя выбрать по какому элементу сортировать и сортируем, используя обычную сортировку пузырьком
    \item Показать все записи структуры:
        Проходимся по каждой записи и выводим ее с помощью функции struct_out
\end{itemize}


\section*{Описание переменных-функция main}
\begin{centering}
\resizebox{14cm}{!}{
    \begin{tabular}{|l|l|l|l|}
        \hline
        \textbf{№} & \textbf{Имя переменной} & \textbf{Тип} & \textbf{Назначение}\\
        \hline
        1 & fp         &FILE*& Указатель на файл\\ 
        \hline
        2 &  filename            & char[]  & Название файла\\ 
        \hline
        3 &  line    &  char[] & Строка структуры \\
        \hline
        4 & BLOCKNOTE        & struct & Структура данных\\ 
        \hline
        5 & month             & int & Введенный пользователем месяц \\
        \hline
        6 & found           & int & Флаг, найден человек или нет \\
        \hline
    \end{tabular}
}
\end{centering}

\newpage

\section*{Вывод}
В данной лабораторной работе изучили работу со структурами данных в языке Си, научились их применять на практике.

\end{document}
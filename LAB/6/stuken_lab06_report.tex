\documentclass[a4paper,12pt]{report}

\usepackage{cmap}
\usepackage[T2A]{fontenc}
\usepackage[utf8]{inputenc}
\usepackage[russian]{babel}
\usepackage{amsmath,amsfonts,amssymb}
\usepackage{graphicx}
\usepackage{sidecap}
\usepackage{wrapfig}
\usepackage{indentfirst}

\begin{document} 

\begin{titlepage} 

\begin{center} 

\large Федеральное государственное автономное образовательное учреждение высшего образования «Санкт-Петербургский государственный электротехнический университет «ЛЭТИ» им. В.И. Ульянова (Ленина)»\\
кафедра Вычислительной техники\\[5cm] 

\huge ОТЧЕТ\\ по лабораторной работе № 6\\[0.5cm] 
\large <<Структуры. Массивы структур>>\\[3.7cm]

\begin{minipage}{1\textwidth}
    \begin{flushleft}
        \emph{Автор:} Стукен В.А.\\
        \emph{Группа:} 2307\\
        \emph{Факультет:} ФКТИ\\
        \emph{Преподаватель:} Аббас Саддам Ахмед\\
    \end{flushleft}
\end{minipage}

\vfill

Санкт-Петербург, 2023\\
{\large \LaTeX}

\end{center}
\thispagestyle{empty}
\end{titlepage}

\section*{Задание(вариант 5)}
Описать структуру с именем NOTE1, содержащую поля: Name – фамилия и инициалы, TELE – номер телефона, DATE – дата рождения (год, месяц, число).

Написать программу, выполняющую:
\begin{itemize}
    \item ввод данных из файла CSV в массив BLOCKNOTE, состоящий из 9 элементов типа NOTE1, записи должны быть упорядочены по инициалам;
    \item вывод на экран информации о людях, чьи дни рождения приходятся на месяц, значение которого введено с клавиатуры, если таких нет – выдать сообщение.

\end{itemize}

\section*{Постановка задачи и описание решения}
\par

Сначала описываем структуру NOTE1 с полями Name,TELE и DATE. 
Далее описываем функцию $cmp$, которая сравнивает две строки из структуры.
В main() открываем файл .csv и читаем построчно каждую его строку и делим ее по запятым, записывая каждое отделенное значение в соотстветствующую ячейку структуры.
Закрываем файл.
Далее сортируем строки полученной структуры по инициалам(поле Name).
Далее просим пользователя ввести номер месяца и выводим данные тех людей, чей день рождения приходится на введенный месяц. 
Если же таких людей нет, то выводим сообщение об этом.

\section*{Описание переменных-функция main}
\begin{centering}
\resizebox{14cm}{!}{
    \begin{tabular}{|l|l|l|l|}
        \hline
        \textbf{№} & \textbf{Имя переменной} & \textbf{Тип} & \textbf{Назначение}\\
        \hline
        1 & fp         &FILE*& Указатель на файл\\ 
        \hline
        2 &  filename            & char[]  & Название файла\\ 
        \hline
        3 &  line    &  char[] & Строка структуры \\
        \hline
        4 & BLOCKNOTE        & struct & Структура данных\\ 
        \hline
        5 & month             & int & Введенный пользователем месяц \\
        \hline
        6 & found           & int & Флаг, найден человек или нет \\
        \hline
    \end{tabular}
}
\end{centering}

\newpage

\section*{Вывод}
В данной лабораторной работе изучили работу со структурами данных в языке Си, научились их применять на практике.

\end{document}